
\chapter{Setting the Stage\label{chap:2}}

In this chapter, we describe the setting in which the theoretical
results of this thesis are stated, and review the existing theory
upon which the original results presented in the following chapters
will build.


\section{Objects and Notation}

We first fix the main objects of study and notation. We will work
on the probability space $\left(\Omega,\mathcal{B}\left(\Omega\right),\mathbb{P}\right)$,
where the sample space $\Omega=\mathcal{C}\left(\left[0,1\right],\mathcal{R}\right)$
is the space of real-valued continuous functions $\omega:\left[0,1\right]\rightarrow\mathcal{R}$,
which is equipped with its Borel $\sigma$-algebra $\mathcal{B}(\Omega)$.
As usual, $\Omega$ is endowed with the canonical filtration $\{\mathcal{F}_{t}\}_{t\in[0,1]}$
generated by the canonical process $X=\left\{ X_{t}\right\} _{t\in\left[0,1\right]}$,
defined as 
\[
X_{t}\left(\omega\right)\coloneqq\omega\left(t\right),
\]
where $\omega\in\Omega$. For any subset $\mathcal{T}\subseteq\left[0,1\right]$,
let $X_{\mathcal{T}}=\left\{ X_{t}\right\} _{t\in\mathcal{T}}$ be
the canonical process at times $\mathcal{T}$ and $\mathbb{P}_{\mathcal{T}}$
be the restriction of $\mathbb{P}$ to $\Omega_{\mathcal{T}}\coloneqq X_{\mathcal{T}}(\Omega)$.
In general, we will take the density of a given measure to refer to
its density with respect to the Lebesgue measure unless otherwise
noted.

In the following chapters, we will study in detail time-homogenous
univariate It� diffusions i.e. continuous and strongly Markovian stochastic
processes (see standard references like \citet[Chapter 5 in][]{oksendal}
for details). In particular, we will study processes $X$ of the form
\[
X_{t}=X_{0}+\int_{0}^{t}\mu(X_{s})ds+\int_{0}^{t}\sigma(X_{s})dW_{s},
\]
for a Wiener process $W$ where the second integral is one in the
sense of It�. As per convention, we will refer to such processes as
solutions to the stochastic differential equation (SDE)
\[
dX_{t}=\mu(X_{t})dt+\sigma(X_{t})dW_{t},
\]
with a given initial condition on $X_{0}$. In all diffusions to be
studied, we will assume that $\mu$ and $\sigma$ are suitably regular,
admitting a weakly unique solution (i.e. all solutions have the same
finite-dimensional distributions) and finite speed measure. Note,
for example, that the latter assumption excludes the Wiener process.
Finally, we assume that any diffusion considered satisfies the balance
condition 
\[
p_{t}(x,y)m(x)=p_{t}(y,x)m(y),
\]
where $p_{t}$ and $m$ are the transition density and density of
the speed measure associated with the diffusion. We refer the reader
to \citet[Section 2.1 in][]{bladt-sorensen-2014} for a detailed discussion
of these conditions.

We will denote probability measures by capital letters $\mathbb{P},\mathbb{Q},\dots$
in blackboard boldface. Random variables will be denoted by capital
letters $X,Y,\dots$, and their particular realizations by lower-case
letters $x,y,\dots$ as per convention. Vectors $\mathbf{x},\mathbf{y},\dots,$
will be denoted in boldface. Often, we will suppress set braces in
subscripts i.e. $X_{0,1}$ will denote the value of $X$ at times
$\{0,1\}$, and we will frequently use $\mathbf{x}^{(i)}$ to denote
the $i$th element of a vector of countable length $\mathbf{x}$ to
avoid the overloading of the subscript operator. Having set our notation,
we now review some existing theory related to stochastic bridges which
will form the foundation of the novel theoretical results to be presented
in the following chapters.


\section{Markov and Generalized Bridges}

In this section, we introduce the notion of a Markov bridge, pointing
the reader towards references like \citet[Section 1 in][]{leonard-2014}
for an in-depth exposition of the subject. Then, we will summarize
the relevant results of \citet{baudoin-2002} with respect to generalized
bridge measures, which formalize the notion of conditioning a process
on the distribution of its values. So-called Baudoin conditionings
will feature heavily in the chapters to follow. None of the theory
discussed in this section is original to this thesis.

First, we briefly review the well-known properties of Markov bridges.
Fix $\mathbb{P}$ as a Markov measure. Under fairly general conditions,
like those laid out in \citet[Proposition 4 in][]{fitz-pitman-1993},
$\mathbb{P}$ can be disintegrated as a mixture of regular conditional
measures, 
\[
\mathbb{P}=\int_{\mathcal{R}^{2}}\mathbb{P}(\cdot\mid X_{0}=x,X_{1}=y)\mathbb{P}_{0,1}(dxdy),
\]
where $\mathbb{P}(\cdot\mid X_{0}=x,X_{1}=y)$ exists for all $x,y$
in the state space and can be constructed using Doob's method of $h$-transforms
(see \citet[Section 45 in][]{rogers} for a treatment of Doob's $h$-transform
for continuous-time processes). We refer to such a measure as the
bridge measure $\mathbb{P}^{x,y}$ of $\mathbb{P}$. Note that the
canonical process on $\mathbb{P}^{x,y}$ can be understood as the
canonical process on $\mathbb{P}$, with its endpoints pinned to certain
values at $t=0$ and $t=1$. It can be easily shown that bridge measures
inherit the Markov property i.e. $\mathbb{P}^{x,y}$ is also a Markov
measure. In this thesis, we will focus on measures induced by It�
diffusions, which are a special case of Markov measures; a diffusion
bridge, then, is a diffusion process with its endpoints fixed at certain
values.

The idea of a bridge measure has been expanded upon significantly
in the literature (see, for instance, \citet{alili-2002,baudoin-coutin-2007},
and \citet{sottinen-yazigi-2014}). A generalization of particular
interest to this thesis was first proposed by \citet{baudoin-2002};
rather than conditioning a process on the value of its endpoints,
\citet{baudoin-2002} formalizes the sense in which arbitrary functionals
of a process can be conditioned on their law. We consolidate and present
some of the major results of Baudoin below, working on the usual probability
space $(\Omega,\mathcal{B}(\Omega),\mathbb{P})$. First, we define
a Baudoin conditioning.
\begin{defn}[{\citet[Definition 1 in][]{baudoin-2002}}]
\label{def:conditioning}A Baudoin $(Y,\nu)$-conditioning on the
usual probability space is a tuple $(Y,\nu)$ where 
\begin{aenumerate}
\item $Y$ is an $\mathcal{F}_{1}$-measurable random variable valued in
$\mathcal{R}^{k}$ representing $k$ functionals of the path being
conditioned and
\item $\nu$ is a law representing the conditioning, 
\end{aenumerate}
such that, if the law of $Y$ under $\mathbb{P}$ is $\mathbb{P}_{Y}$,
\begin{aenumerate}
\item $\nu$ is absolutely continuous with respect to $\mathbb{P}_{Y}$
and
\item for $t\in[0,1)$ and $y\in\mathcal{R}^{k}$, there exists a process
$\eta_{t}^{y}$ such that for any $\mathcal{F}_{t}$-measurable and
bounded random variable $Z$, $\mathbb{E}_{\mathbb{P}}\left[Z\mid Y=y\right]=\mathbb{E}_{\mathbb{P}}\left[\eta_{t}^{y}Z\right]$.
\end{aenumerate}
\end{defn}
From this point on, we will always assume that the tuples $(Y,\nu)$
we encounter satisfy the conditions above. The next result demonstrates
the existence and uniqueness of a measure which satisfies intuitive
notions of what it means to be conditioned on the law of a random
variable. Moreover, it says that such a conditioned measure is the
convex diversity minimizing measure with respect to a family of related
probability measures.
\begin{prop}[{\citet[Proposition 3 and 6 in][]{baudoin-2002}}]
\label{prop:pv-1}Fix a Baudoin conditioning $(Y,\nu)$ on the usual
probability space. Then, there exists a unique probability measure
$\mathbb{P}^{\nu}$ such that 
\begin{aenumerate}
\item for any bounded random variable $Z:\Omega\rightarrow\mathcal{R}$,
$\mathbb{E}_{\mathbb{P}^{\nu}}[Z\mid Y]=\mathbb{E}_{\mathbb{P}}[Z\mid Y]$
and
\item the law of $Y$ under $\mathbb{P}^{\nu}$ is $\nu$.
\end{aenumerate}
In particular, for any convex function $\varphi:\mathcal{R}^{+}\rightarrow\mathcal{R}$,
if $\mathbb{E}[|\varphi(d\nu/d\mathbb{P}_{Y})|]$ is bounded, then
$\mathbb{P}^{\nu}$ satisfies
\[
\inf_{\mathbb{Q}\in\mathcal{Q}_{\mathbb{P},\nu,\varphi}}\mathbb{E}_{\mathbb{Q}}\left[\varphi\left(\frac{d\mathbb{Q}}{d\mathbb{P}}\right)\right]=\mathbb{E}_{\mathbb{P}^{\nu}}\left[\varphi\left(\frac{d\mathbb{P}^{\nu}}{d\mathbb{P}}\right)\right],
\]
where $\mathcal{Q}_{\mathbb{P},\nu,\varphi}$ is the set of measures
such that for any $\mathbb{Q}\in\mathcal{Q}_{\mathbb{P},\nu,\varphi}$,
\begin{aenumerate}
\item $\mathbb{Q}$ is absolutely continuous with respect to $\mathbb{P}$,
\item $\mathbb{E}\left[|\varphi\left(d\mathbb{Q}/d\mathbb{P}\right)|\right]$
is bounded, and
\item the law of $Y$ under $\mathbb{Q}$ is $\nu$.
\end{aenumerate}
\end{prop}
We refer to $\mathbb{P}^{\nu}$ as the Baudoin $(Y,\nu)$-conditioning
of $\mathbb{P}$, and the canonical process on $\mathbb{P}^{\nu}$
as the Baudoin $(Y,\nu)$-bridge of the canonical process on $\mathbb{P}$.
\propref{pv-1} is a central result, and shows that of the measures
which are sufficiently similar to $\mathbb{P}$ and under which the
law of $Y$ is $\nu$, the convex diversity minimizing measure is
precisely the measure which coincides with $\mathbb{P}$ on events
independent of $Y$. 

While this characterization is theoretically pleasing, it is not practically
useful in working with $\mathbb{P}^{\nu}$. Fortunately, Baudoin points
out that $\mathbb{P}^{\nu}$ can be disintegrated along the $\sigma$-algebra
generated by $Y$ as
\[
\mathbb{P}^{\nu}=\int_{\mathcal{R}^{k}}\mathbb{P}(\cdot\mid Y=y)\nu(dy),
\]
since the regular conditional probability $\mathbb{P}(\cdot\mid Y=y)$
exists by the existence of $\eta_{t}^{y}$. This observation allows
us to uniquely construct $\mathbb{P}^{\nu}$ when $\eta_{t}^{y}$
exists (which we always assume going forward). In the language of
the discussion regarding Markov bridges above, $\mathbb{P}^{\nu}$
is a mixture of the regular conditional probabilities $\mathbb{P}(\cdot\mid Y=y)$
with respect to some mixing measure $\nu$. Note that by definition,
under $\mathbb{P}^{\nu}$, $Y$ has law $\nu$, and as such $\mathbb{P}^{\nu}=\mathbb{P}$
if and only if $\nu=\mathbb{P}_{Y}$.

With the necessary theoretical foundations set, we move now to the
original portion of this thesis, and formalize the problem of inference
on distributional data.
