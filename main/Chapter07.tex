
\chapter{Conclusions and Future Directions\label{chap:7}}

We conclude by briefly reviewing the contributions made and suggesting
future directions for research.

This thesis considers the novel (to the best of our knowledge) problem
of statistical inference when given the distribution, instead of particular
observations, of a stochastic process at discrete points in time.
This is a generalization of the classical problem of inference on
discrete observations that has been well-studied in the literature.
We develop novel inference strategies when given distributional data
and the first samplers for a particular class of generalized bridges.
Simulations and empirical studies demonstrate the correctness of the
methods described and their broad applicability.

Future directions for research abound, especially given the novel
problem of this thesis. We highlight some potential lines of exploration
here. Most pressingly, a formal investigation into the properties
of the imputation scheme proposed in \chapref{6} is needed to justify
its use. It seems intuitive that a model which minimizes its K-L divergence
from a true but unknown model, conditioned on the available distributional
data from the true model, is our best guess at the underlying truth;
however, no attempt is made in this thesis to formalize this thought.
Since we imagine most applications of the methods presented in this
paper will involve distributional data and a stochastic model specified
only up to its parameters, rigorously justifying the proposed imputation
scheme is of utmost importance.

Of equal urgency is a rigorous characterization of the properties
of the MKLDE. This would complement the mathematical remarks made
in \chapref{EM} and set the results of this thesis on firmer theoretical
ground. Such characterizations ought to come fairly easily from existing
results, given the close links between MKLDE and MLE.

A third avenue of research comes from the simulation results presented
in \chapref{5}, showing that the generalized bridge samplers appeared
to be empirically robust against the relaxation of a variety of assumptions
in their derivation. We conjecture based on simulations for Wiener
bridges that in particular, the restriction of diffusions to those
with finite speed measure considered in this thesis is overly strict.
Perhaps a sampler (potentially more ``approximate'', in some sense,
than the sampler proposed here) for more general diffusions is not
too far from reach.

Empirically, an area ripe for an application of the methods proposed
in this thesis is the modeling of asset prices based on distributional
data from options markets: Having access to the market-determined
distribution of prices at arbitrary horizons would be incredibly valuable
for practitioners and academics alike. Notwithstanding the trading
strategies that could immediately result, evaluating the efficiency
of options markets in pricing the distributional features of asset
prices over time would be a particular area of academic interest.
