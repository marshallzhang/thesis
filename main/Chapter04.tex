
\chapter{The Simulation of Generalized Bridges\label{chap:4}}

In this chapter, we develop methods to sample approximate and exact
generalized bridges. 


\section{$\nu-$bridges}

First, we introduce the notion of a $\nu$-bridge. Consider the solution
$X$ to the stochastic differential equation
\begin{equation}
dX_{t}=\mu(X_{t})dt+\sigma(X_{t})dW_{t},\label{eq:basic-sde}
\end{equation}
satisfying the usual assumptions, where $\mu$ and $\sigma$ are fully
specified (this is unlike (\ref{eq:sde-with-theta}) in the previous
chapter, where the drift and diffusion were only known up to some
parameters) and $W$ is a Wiener process. Suppose $X$ induces the
probability measure $\mathbb{Q}$ on the usual probability space,
fix $\mathcal{T}\subseteq[0,1]$ as a set of times, and let $\nu$
be a distribution that is absolutely continuous to the law of $X_{\mathcal{T}}$
under $\mathbb{Q}$. Then, we can understand the Baudoin $(X_{\mathcal{T}},\nu)$-conditioning
of $\mathbb{Q}$ as a generalized diffusion bridge measure, in the
sense that the canonical process on $\mathbb{Q}^{\nu}$ is a regular
diffusion bridge when $\nu$ is the product of Dirac delta measures.\footnote{This statement is more precisely made in some limiting sense, since
the product measure of Dirac delta measures is not absolutely continuous
to $\mathbb{Q}_{\mathcal{T}}$ when $X$ is an It� diffusion. However,
the product measure of Dirac delta measures can be the limit of a
sequence of probability measures which satisfy such a condition.} Since $\mathbb{Q}$ is Markovian, we may, without loss of generality,
set $\mathcal{T}=\{0,1\}$ and restrict our attention to Baudoin conditionings
of the form $(X_{0,1},\nu)$. Then, a $\nu$-bridge of $X$ is defined
as follows.
\begin{defn}
Let $\nu_{0}$ be the marginal distribution of $\nu$ at $t=0$. Then,
let $X$ be a solution to (\ref{eq:basic-sde}) with $X_{0}\sim\nu_{0}$,
and let $\mathbb{Q}$ be the measure it induces. The $\nu-$bridge
of $X$ is the canonical process under the Baudoin $(X_{0,1},\nu)-$conditioning
of $\mathbb{Q}$, denoted $\mathbb{Q}^{\nu}$.
\end{defn}
Baudoin bridges are well-characterized in highly general settings;
we therefore defer to \citet{baudoin-2002} for a theoretical treatment
of their properties. In this chapter, we instead focus on simulating
$\nu$-bridges, motivated by our MCEM scheme's need to sample from
paths under Baudoin bridge measures.


\section{Approximate Simulation of $\nu$-Bridges}

In this section, we propose a sampler of approximate $\nu$-bridges
and make precise the notion of what it means to be an ``approximate''
bridge.

A substantial literature surrounds the simulation of diffusion bridges
since na�ve simulation attempts will have a unacceptably high rejection
probability when attempting to hit the desired endpoint (which, after
all, is a measure zero event). One can imagine that simulating a $\nu-$bridge
should also be non-trivial, since we are not simply conditioning on
the measure zero event of $\left\{ X_{1}=x_{1}\right\} $, but rather
the ``event'' that $X_{0}$ and $X_{1}$ follow some joint distribution.

An overview of the recent literature in simulating traditional diffusion
bridges can be found in \citet{papa-roberts-2012}. A particular method
of interest is that of \citet{bladt-sorensen-2014}, where an intuitively
simple and computationally inexpensive procedure is used to simulate
approximate diffusion bridges. In this section, we generalize the
theorem of \citet[Theorem 2.1 in][]{bladt-sorensen-2014} to an original
result for the simulation of $\nu-$bridges.

The general idea of the proposed sampler is to simulate two diffusion
processes, with their initial conditions jointly distributed according
to $\nu$. If the second process, when reversed in time, intersects
with the first, we may weld together the processes (one running forward
in time, the other running backward) into a diffusion with endpoints
distributed according to $\nu$. This intuitive construction is visualized
in \figref{bridge-intuit} and formalized in \thmref{approx-sim}. 

\begin{figure}[t]
\includegraphics[width=1\columnwidth]{/Users/marshall/Documents/senior/thesis/figures/gen_bridges}

\caption{\label{fig:bridge-intuit}A visualization of the approximate $\nu$-bridge
simulation strategy we propose in this chapter. The left graphic depicts
two diffusion processes with some joint initial distribution (whose
marginals are depicted). One process is simulated forward in time
and other is simulated backward in time. The right graphic depicts
how we can join the two processes (now in red) to create an approximate
$\nu$-bridge (in black).}
\end{figure}


First, we state a lemma of \citet{bladt-sorensen-2014} which will
help characterize the behavior of a time-reversed process.
\begin{lem}[{\citet[Lemma 2.2 in][]{bladt-sorensen-2014}}]
\label{lem:bladt-lemma}Consider a solution $X$ to \eqref{basic-sde}
with initial condition $X_{0}=b$, and let $X^{\leftarrow}$ be the
time-reversed process $X_{t}^{\leftarrow}=X_{1-t}$. Then, $X^{\leftarrow}$
is equal in distribution to a stationary solution $X^{*}$ to (\ref{eq:basic-sde})
conditional on $X_{1}^{*}=b$. Equivalently, $X^{\leftarrow}$ is
equal in distribution to a solution to (\ref{eq:basic-sde}) with
initial condition $X_{0}\sim p_{1}(b,\cdot)$ where $p_{t}$ is the
associated transition kernel.
\end{lem}
With \lemref{bladt-lemma} in hand, we can now prove the central result
of this chapter, which constructs an approximate $\nu$-bridge of
$X$ from unconstrained versions of $X$, and makes precise the sense
in which such a bridge is approximate. For expository convenience,
we refer to a vector of $m$ independent solutions to (\ref{eq:basic-sde})
as an $m$-dimensional solution to the same, and set $\inf\varnothing\coloneqq\infty$.
Furthermore, recall that we denote the $i$th element of the vector
$\mathbf{X}$ with $\mathbf{X}^{(i)}$.
\begin{thm}
\label{thm:approx-sim}Let $\mathbf{X}^{\nu}$ be a $2$-dimensional
solution to (\ref{eq:basic-sde}) with initial condition $\mathbf{X}^{\nu}\sim\nu$.
Let $\tau=\inf\{t\in[0,1)\mid\mathbf{X}_{t}^{\nu,\left(1\right)}=\mathbf{X}_{1-t}^{\nu,\left(2\right)}\}$.
Define a process $Z$ as 
\[
Z_{t}=\begin{cases}
\mathbf{X}_{t}^{\nu,\left(1\right)} & \mbox{if }0\leq t\leq\tau\\
\mathbf{X}_{1-t}^{\nu,\left(2\right)} & \mbox{if }\tau<t\leq1
\end{cases}
\]
on the event $\{\tau<1\}$ and $Z=\mathbf{X}^{\nu,\left(1\right)}$
otherwise.

Then, the measure induced by $Z$ conditional on $\{\tau<1\}$ is
equal to the measure induced by a $\nu-$bridge of a solution to (\ref{eq:basic-sde}),
conditional on such a bridge being hit by an independent stationary
solution to (\ref{eq:basic-sde}) with initial distribution $p_{1}(Z_{1},\cdot)$.\end{thm}
\begin{proof}
Let $\mathbf{X}^{*}$ be an independent $2$-dimensional stationary
solution to (\ref{eq:basic-sde}). We begin by defining two functionals
of $\mathbf{X}^{\nu}$ and $\mathbf{X}^{*}$.

First, we set the stopping times $\rho^{(i)}=\inf\{t\in[0,1)\mid\mathbf{X}_{t}^{\nu,(i)}=\mathbf{X}_{t}^{*,(i)}\}$,
and define a stochastic process $\mathbf{Y}$ such that 
\[
\mathbf{Y}_{t}^{(i)}=\begin{cases}
\mathbf{X}_{t}^{\nu,(i)} & \mbox{if }0\leq t\leq\rho^{(i)}\\
\mathbf{X}_{t}^{*,(i)} & \mbox{if }\rho^{\left(i\right)}<t\leq1
\end{cases}
\]
on the event $\mathscr{R}^{(i)}=\{\rho^{(i)}<1\}$ and $\mathbf{Y}^{(i)}=\mathbf{X}^{\nu,(i)}$
otherwise. Let $\mathscr{R}=\mathscr{R}^{(1)}\cup\mathscr{R}^{(2)}$,
and note that the coordinates of $\mathbf{Y}$ are mutually independent.
Let $\mathbb{Y}$ be the measure induced by $\mathbf{Y}$.

Similarly, we set the stopping times $\tau^{(i)}=\inf\{t\in[0,1)\mid\mathbf{X}_{t}^{\nu,(i)}=\mathbf{X}_{1-t}^{\nu,(\setminus i)}\}$,
and define a stochastic process $\mathbf{Z}$ such that
\[
\mathbf{Z}_{t}^{(i)}=\begin{cases}
\mathbf{X}_{t}^{\nu,(i)} & \mbox{if }0\leq t\leq\tau^{(i)}\\
\mathbf{X}_{1-t}^{\nu,(\setminus i)} & \mbox{if }\tau^{(i)}<t\leq1
\end{cases}
\]
on the event $\mathscr{T}^{(i)}=\{\tau^{(i)}<1\}$ and $\mathbf{Z}^{(i)}=\mathbf{X}^{\nu,(1)}$
otherwise. Let $\mathscr{T}=\mathscr{T}^{(1)}\cup\mathscr{T}^{(2)}$,
and denote the measure induced by $\mathbf{Z}$ as $\mathbb{Z}$.
For notational ease throughout this proof, for any pair $\mathbf{p}=(\mathbf{p}^{(1)},\mathbf{p}^{(2)})$,
we define $\overleftrightarrow{\mathbf{p}}=(\mathbf{p}^{(2)},\mathbf{p}^{(1)})$.

Now, let $\mathscr{B}$ be the event that $\mathbf{Y}_{1}=\overleftrightarrow{\mathbf{Y}_{0}}$
i.e. the event $\{\omega\in\Omega\mid\mathbf{Y}_{1}(\omega)=\overleftrightarrow{\mathbf{Y}_{0}(\omega)}\}$.

We first show that $\mathbb{Y}\left(\cdot\mid\mathscr{B}\cap\mathscr{R}\right)$
is equal to $\mathbb{Z}\left(\cdot\mid\mathscr{T}\right)$. By definition
of $\mathbf{Y}$, we know that $\mathscr{B}$, conditional on $\mathscr{R}$,
is the event that $\mathbf{X}_{1}^{*}=\overleftrightarrow{\mathbf{X}_{0}^{\nu}}$.
Then, by \lemref{bladt-lemma}, $\mathbf{X}^{*}$ conditional on $\mathscr{B}$
is equal in distribution to the time-reversed process $\overleftrightarrow{\mathbf{X}_{1-t}^{\nu}}$,
which by construction of $\mathbf{Z}$ and the strong Markov property
gives that $\mathbb{Y}\left(\cdot\mid\mathscr{B}\cap\mathscr{R}\right)=\mathbb{Z}\left(\cdot\mid\mathscr{T}\right)$.

We now show that $\mathbb{Y}\left(\cdot\mid\mathscr{B}\cap\mathscr{R}\right)$
is also equal to the measure induced by two related $\nu$-bridges
of $X$, subject to some conditions. In particular, we note that we
may disintegrate $\mathbb{Y}(\cdot\mid\mathscr{B}\cap\mathscr{R})$
by writing
\begin{align}
\mathbb{Y}(\cdot\mid\mathscr{B}\cap\mathscr{R}) & =\int\mathbb{Y}(\cdot\mid\mathscr{R},\mathbf{Y}_{1}=\overleftrightarrow{\mathbf{Y}_{0}},\mathbf{Y}_{0}=\mathbf{y}_{0})\nu(d\mathbf{y}_{0})\nonumber \\
 & =\int\mathbb{Y}(\cdot\mid\mathscr{R},\mathbf{Y}_{0,1}^{(2)}=\overleftrightarrow{\mathbf{Y}_{0,1}^{(1)}},\mathbf{Y}_{0,1}^{(1)}=\mathbf{y}_{0,1}^{(1)})\nu(d\mathbf{y}_{0,1}^{(1)}).\label{eq:disintY}
\end{align}
We arrive at the the first equality by disintegrating $\mathbb{Y}$
along the $\sigma$-algebra generated by $\mathbf{Y}_{0}$ and conditioning
on $\mathscr{B}$. To arrive at the second equality, we merely need
to note that for any $\omega\in\Omega$, if $\mathbf{Y}_{1}^{(1)}(\omega)$
is set to equal $\mathbf{Y}_{0}^{(2)}(\omega)$, then $\mathbf{Y}_{0}(\omega)=\mathbf{Y}_{0,1}^{(1)}(\omega)$
by definition (the same argument holds for $\mathbf{Y}_{1}$); intuitively,
if $\mathbf{Y}_{1}^{(1)}$ is equal to $\mathbf{Y}_{0}^{(2)}$, the
joint distribution of $\mathbf{Y}_{0}$ should be the same as the
joint distribution of the endpoints of $\mathbf{Y}^{(1)}$.

(\ref{eq:disintY}) reveals that we can understand $\mathbb{Y}(\cdot\mid\mathscr{B}\cap\mathscr{R})$
as the Baudoin $(\mathbf{Y}_{0,1}^{(1)},\nu$)-conditioning of $\mathbb{Y}$,
conditional on $\mathbf{Y}_{0,1}^{(2)}=(\mathbf{Y}_{1}^{(1)},\mathbf{Y}_{0}^{(1)})$
and the coordinates of $\mathbf{Y}$ being hit by the respective coordinates
of $\mathbf{X}^{*}$. Then, $\mathbb{Y}(\cdot\mid\mathscr{B}\cap\mathscr{R})$
is the measure induced by a $\nu-$bridge of $X$ and a related, time-
and coordinate-reversed $\nu$-bridge of $X$, where each of the bridges
is hit by the respective coordinate of an independent diffusion $\mathbf{X}^{*}$
conditioned such that $\mathbf{X}_{1}^{*}=\mathbf{Y}_{1}$.

To complete the proof, it suffices to demonstrate that the restriction
of $\mathbb{Y}\left(\cdot\mid\mathscr{B}\cap\mathscr{R}\right)$ to
its first coordinate is a $\nu-$bridge of $X$ subject to the appropriate
conditions, since the first coordinate of $\mathbf{Z}$ is equal to
$Z$ by definition.

Since the coordinates of $\mathbf{X}^{*}$ and $\mathbf{Y}$ are all
mutually independent, we integrate over the second coordinate to find
that $\mathbb{Y}^{\left(1\right)}\left(\cdot\mid\mathscr{B}\cap\mathscr{R}\right)$
is the measure induced by a $\nu-$bridge of $X$, conditional on
the bridge being hit by an independent stationary solution $X^{*}$
to (\ref{eq:basic-sde}) with $X_{1}^{*}=b$, where $b$ is the value
of the bridge at $t=1$. By \lemref{bladt-lemma}, $X^{*}$ is therefore
an independent solution to (\ref{eq:basic-sde}) with initial distribution
$p_{1}(b,\cdot)$, and the theorem follows.
\end{proof}
\thmref{approx-sim} says that $Z$ is an approximation to a $\nu-$bridge
of $X$ in the sense that if $\tau<1$, $Z$ is a $\nu-$bridge of
$X$ conditional on $Z$ being hit by an independent solution to (\ref{eq:basic-sde})
with a particular initial distribution (we call the latter process
a hitting diffusion). This implies that the higher the probability
of the bridge and a hitting diffusion intersecting, the closer $Z$
will be to being an exact $\nu-$bridge. 

\thmref{approx-sim} is a natural generalization of the main result
of \citet[Theorem 2.1 in][]{bladt-sorensen-2014}. In particular,
it says that an approximate $\nu$-bridge is simply an approximate
diffusion bridge between $(a,b)\sim\nu$. We can therefore use \thmref{approx-sim}
to simulate approximate $\nu$-bridges by directly extending the approximate
diffusion bridge sampler proposed in \citet{bladt-sorensen-2014}.
Suppose we have a method \noun{diffusion(}\textbf{\noun{$x_{0},\mu,\sigma,N$)}}\noun{
}for simulating observations at $\mathcal{T}=\{0,1/N,\dots,1\}$ of
a solution to (\ref{eq:basic-sde}) with initial condition $X_{0}=x_{0}$
(see \citet[Chapter 10 in][]{kloeden-1992} for examples of such methods).
We define $\delta\coloneqq1/N$, and a crossing of the discrete series
of observations $\mathbf{X}=\{\mathbf{X}_{t}\}_{t\in\mathcal{T}}$
and $\mathbf{Y}=\{\mathbf{Y}_{t}\}_{t\in\mathcal{T}}$ as the event
that there exists an $t\in\mathcal{T}$ such that
\begin{aenumerate}
\item $\mathbf{X}_{t}\geq\mathbf{Y}_{1-t}$ and $\mathbf{X}_{t+\delta}\leq\mathbf{Y}_{1-t-\delta}$,
or
\item $\mathbf{X}_{t}\leq\mathbf{Y}_{1-t}$ and $\mathbf{X}_{t+\delta}\geq\mathbf{Y}_{1-t-\delta}$.
\end{aenumerate}
The resulting approximate $\nu$-bridge sampler is presented in \algoref{approx-sim}.

\begin{algorithm}[tb]
\caption{Approximate $\nu-$bridge Sampler}


\begin{algorithmic}[1] 
\Function{approximate $\nu-$bridge}{$\mu, \sigma, N$} 
\State{\textbf{draw} $(a,b) \sim \nu$}
\State{$\mathbf{X}^{(1)} \gets$ \Call{diffusion}{$a, \mu, \sigma, N$}}
\State{$\mathbf{X}^{(2)} \gets$ \Call{reverse}{\Call{diffusion}{$b, \mu, \sigma, N$}}}
\While{$\mathbf{X}^{(1)}$ and  $\mathbf{X}^{(2)}$ do not cross} 
\State{\textbf{draw} $(a,b)$ from $\nu$}
\State{$\mathbf{X}^{(1)} \gets$ \Call{diffusion}{$a, \mu, \sigma, N$}}
\State{$\mathbf{X}^{(2)} \gets$ \Call{reverse}{\Call{diffusion}{$b, \mu, \sigma, N$}}}
\EndWhile

\If{$\mathbf{X}^{(1)}_0 > \mathbf{X}^{(2)}_1$}
\State{$\tau \gets \min \{t\in \mathcal{T} \mid \mathbf{X}^{(1)}_{t} \leq \mathbf{X}^{(2)}_{t}\}$}
\Else
\State{$\tau \gets \min \{t\in \mathcal{T} \mid \mathbf{X}^{(1)}_{t} \geq \mathbf{X}^{(2)}_{t}\}$}
\EndIf

\State{$Z \gets \begin{cases} \mathbf{X}^{(1)}_{t}, \text{ for } t=0,1/N,\dots,\tau - 1/N \\ \mathbf{X}^{(2)}_{t}, \text{ for } i=\tau,\dots,1 \end{cases}$}

\State \Return $Z$
\EndFunction 
\end{algorithmic}

\label{algo:approx-sim}
\end{algorithm}



\section{Exact Simulation of $\nu-$bridges}

In \chapref{5}, we will present some evidence that the approximate
simulation scheme proposed above produces samples that are qualitatively
similar to exact samples in a variety of situations. However, an exact
simulation scheme is still desirable. In this section, we outline
the method by which \citet{bladt-sorensen-2014} extend their approximate
diffusion bridge sampler to an exact sampler, and then present an
easy modification of this strategy to our approximate $\nu$-bridge
sampler.

We begin with a brief review of \citet[Section 2.3 in][]{bladt-sorensen-2014}'s
derivation of an exact diffusion bridge sampler. First, note if
\begin{aenumerate}
\item $\hat{f}(x)$ is the density of an approximate diffusion bridge, 
\item $f(x)$ is the density of an exact diffusion bridge, and 
\item $I_{x,y}$ is the event that $x,y\in\Omega$ intersect i.e. there
exists some $t$ such that $x_{t}=y_{t}$, then
\end{aenumerate}
\[
\hat{f}(x)=f(x)\frac{\mbox{Pr}(I_{x,Y})}{\mbox{Pr}(I_{X,Y})},
\]


where $Y$ is a hitting diffusion. This fact is proven in \citet[below Theorem 2.1. in][]{bladt-sorensen-2014}.
As such, the acceptance ratio in an Metropolis-Hastings (M-H) scheme
to sample exact diffusion bridges using approximate diffusion bridge
proposals should be 
\begin{equation}
\alpha(X^{(i-1)},X^{'})\coloneqq\frac{\mbox{Pr}(I_{X^{(i-1)},Y})}{\mbox{Pr}(I_{X^{'},Y})},\label{eq:acceptance}
\end{equation}
where $X^{(i-1)}$ is an approximate diffusion bridge and $X'$ is
an independent approximate diffusion bridge proposal. The intuition
for this ratio is as follows: First, as \citet[Theorem 2.1 in][]{bladt-sorensen-2014}
show, an approximate diffusion bridge is a diffusion bridge conditioned
on intersecting with a hitting diffusion (this is the result that
\thmref{approx-sim} generalizes). An M-H scheme with $\alpha$ as
defined in (\ref{eq:acceptance}) accepts, with high probability,
proposals that are not likely to intersect with a hitting diffusion.
As such, the M-H acceptance step effectively counteracts the proliferation
of diffusion bridges that are likely to intersect with hitting diffusions
when using an approximate sampler.

While seemingly simple, (\ref{eq:acceptance}) is in fact unavailable
analytically. Leveraging a result of \citet[Section 2.2 in][]{andrieu-roberts},
\citet{bladt-sorensen-2014} instead use unbiased estimates of $\mbox{Pr}(I_{x,Y})$
to compute an estimate for the acceptance ratio $\alpha$ to develop
their exact diffusion bridge sampler.

\begin{algorithm}[t]
\caption{Exact $\nu-$bridge Sampler}


\begin{algorithmic}[1] 
\Function{$\hat{\rho}$}{$x,M$}
\For{$i=1,\dots,M$}
\State{$T_{i} \gets 1$}
\While{\Call{hitting}{$x_{1}$} and $x$ do not cross}
\State{$T_{i} \gets T_{i}+1$}
\EndWhile
\EndFor
\State \Return{\Call{mean}{$T_i$}}
\EndFunction

\Function{exact $\nu$-bridge}{$\mu, \sigma, N, M$}
\State{$\hat{\mathbf{X}}^{(1)} \gets$ \Call{approximate $\nu$-bridge}{$\mu, \sigma, \nu, N$}}
\State{$\hat{\mathbf{X}}^{(2)} \gets$ \Call{approximate diffusion bridge}{$\hat{\mathbf{X}}^{(1)}_0, \hat{\mathbf{X}}^{(1)}_1, \mu, \sigma, N$}}
\State{$\alpha \gets \min{\left\{1,\cfrac{\hat{\rho}(\hat{\mathbf{X}}^{(2)},M)}{\hat{\rho}(\hat{\mathbf{X}}^{(1)},M)}\right\}}$}
\State \Return $\hat{\mathbf{X}}^{(2)}$ with probability $\alpha$, $\hat{\mathbf{X}}^{(1)}$ otherwise
\EndFunction 


\end{algorithmic}

\label{algo:exact-sim}
\end{algorithm}


Now, we present an adaptation of this strategy to develop an exact
$\nu$-bridge sampler. We use the observation that our scheme for
sampling an approximate $\nu$-bridge can be reduced to
\begin{aenumerate}
\item drawing $(a,b)\sim\nu$, then 
\item sampling an approximate diffusion bridge from $a$ to $b$ using \citet{bladt-sorensen-2014}'s
approximate sampler. 
\end{aenumerate}
Sampling an exact $\nu$-bridge then, can be achieved simply by drawing
$(a,b)\sim\nu$, then sampling an exact diffusion bridge from $a$
to $b$ using \citet{bladt-sorensen-2014}'s exact sampler. We assume
we have the methods
\begin{aenumerate}
\item \noun{hitting}($b$), which returns discrete observations from a solution
to (\ref{eq:basic-sde}) with initial condition $p_{1}(b,\cdot)$,
and 
\item \noun{approximate diffusion bridge($a,b,\mu,\sigma,N)$,} which is
the approximate sampler of \citet{bladt-sorensen-2014} for a diffusion
bridge from $a$ to $b$.
\end{aenumerate}
\noun{hitting}($b$) is simple to implement due to the observation
that a diffusion on $[0,1]$ with initial condition $p_{1}(b,\cdot)$
is equal in distribution to the same diffusion over $[1,2]$ with
initial condition $b$. As outlined in \citet[Section 2.3 in][]{bladt-sorensen-2014},
the function $\hat{\rho}\left(x,M\right)$ in \algoref{exact-sim}
is an unbiased Monte Carlo estimate for the expectation of a geometric
distribution with probability of success $\mbox{Pr}(I_{x,Y})$ i.e.
an unbiased estimate for $1/\mbox{Pr}(I_{x,Y})$.

With\noun{ exact $\nu-$bridge}, we end the original theoretical portion
of this thesis. \algoref{exact-sim} is a sampler of the canonical
process on a Baudoin $(X_{0,1},\nu)$-conditioning of $\mathbb{Q}$.
This algorithm, after some trivial generalizations (for instance,
simulating a path over intervals other than $[0,1]$), is precisely
the method we need to implement the MCEM algorithm proposed in \algoref{mcem}.
It thereby completes our answer to the question of inference on distributional
data, allowing us to sample from an arbitrary Baudoin bridge measure.
Furthermore, it answers in full the question of imputation when given
distributional data and a fully specified stochastic process: we may
simply simulate the process between the distributional data on which
it is conditioned. In the next two chapters, we put theory to practice
as we evaluate and apply the methods developed thus far in a variety
of simulation and empirical settings.
