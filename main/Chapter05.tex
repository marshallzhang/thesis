%% LyX 2.1.4 created this file.  For more info, see http://www.lyx.org/.
%% Do not edit unless you really know what you are doing.
\documentclass[american,fontsize=11pt,paper=a4,twoside,openright,titlepage,numbers=noenddot,headinclude,BCOR=5mm,footinclude=true,cleardoublepage=empty]{scrreprt}
\usepackage[T1]{fontenc}
\setcounter{secnumdepth}{2}
\usepackage{array}
\usepackage{refstyle}
\usepackage{rotfloat}
\usepackage{multirow}
\usepackage{amsmath}
\usepackage{amsthm}
\usepackage{graphicx}
\usepackage[numbers]{natbib}

\makeatletter

%%%%%%%%%%%%%%%%%%%%%%%%%%%%%% LyX specific LaTeX commands.

\AtBeginDocument{\providecommand\partref[1]{\ref{part:#1}}}
\AtBeginDocument{\providecommand\chapref[1]{\ref{chap:#1}}}
\AtBeginDocument{\providecommand\figref[1]{\ref{fig:#1}}}
\AtBeginDocument{\providecommand\tabref[1]{\ref{tab:#1}}}
%% Because html converters don't know tabularnewline
\providecommand{\tabularnewline}{\\}
\RS@ifundefined{subref}
  {\def\RSsubtxt{section~}\newref{sub}{name = \RSsubtxt}}
  {}
\RS@ifundefined{thmref}
  {\def\RSthmtxt{theorem~}\newref{thm}{name = \RSthmtxt}}
  {}
\RS@ifundefined{lemref}
  {\def\RSlemtxt{lemma~}\newref{lem}{name = \RSlemtxt}}
  {}


%%%%%%%%%%%%%%%%%%%%%%%%%%%%%% Textclass specific LaTeX commands.
% Classic Thesis Style loader
\makeatother
\input{classicthesis-config.tex}
\makeatletter
% use Latin Modern instead of Computer Modern sans serif
\renewcommand{\sfdefault}{lmss}

%%%%%%%%%%%%%%%%%%%%%%%%%%%%%% User specified LaTeX commands.
\usepackage{algorithm,algpseudocode}
\newref{prop}{name=Proposition~,Name=Proposition~}
\newref{prob}{name=Problem~,Name=Problem~}
\newref{thm}{name=Theorem~,Name=Theorem~}
\newref{chap}{name=Chapter~,Name=Chapter~}
\newref{part}{name=Part~,Name=Part~}
\newref{tab}{name=Table~,Name=Table~}
\newref{algo}{name=Algorithm~,Name=Algorithm~}
\newref{lem}{name=Lemma~,Name=Lemma~}
\newref{fig}{name=Figure~,Name=Figure~}
\usepackage{array}
\usepackage{booktabs}
\usepackage[margin=10pt,font=small,labelfont=bf,labelsep=endash]{caption}

\makeatother

\usepackage{babel}
\begin{document}

\chapter{Simulation Studies\label{chap:5}}

In this chapter, we study the methods proposed in \partref{2} in
a controlled simulation environment to demonstrate their correctness.
First, we demonstrate the correctness of the generalized bridge sampling
scheme proposed in \chapref{4}, and show that the approximate scheme
produces samples of generalized bridges that are extremely close in
distribution to the exact sampler. Then, with a correct sampler in
hand, we demonstrate that the MCEM scheme proposed in \chapref{EM}
converges to the Kullback-Liebler divergence minimizing parameter
in a variety of situations as expected.


\section{Objects and Methods}

As usual, we begin by fixing the objects of study. We shall consider
two examples of diffusions which satisfy the assumptions outlined
in \chapref{2}. The first process is an Ornstein-Uhlenbeck (OU) process,
which is a solution to the SDE 
\begin{equation}
dO_{t}=-\lambda\left(O_{t}-\mu\right)dt+\sigma dW_{t},\label{eq:ou-sde}
\end{equation}
for some initial condition on $O_{0}$ and a Wiener process $W_{t}$.
OU processes are both theoretically appealing (representing the only
non-trivial stationary, Gaussian, and Markovian process) and practically
useful (used to model everything from springs to currency exchange
rates\textemdash see standard physics references or \citet{hirsa-2012}
for applications in finance). The second process we shall focus on
here is a Cox-Ingersoll-Ross (CIR) process, which is a solution to
the SDE 
\begin{equation}
dC_{t}=-\lambda\left(C_{t}-\mu\right)dt+\sigma\sqrt{C_{t}}dW_{t},\label{eq:cir-sde}
\end{equation}
with some initial condition on $C_{0}$ and a Wiener process $W_{t}$.
We assume as per the literature that $2\theta\mu\geq\sigma^{2}$,
so $C$ lives on the positive real line. With this restriction, CIR
processes form the foundation of common short-term interest rate and
stochastic volatility models (see \citet{cir-1985} and \citet{heston-1993}).

In general, we will refer to the parameter vector of either process
as $\theta$, with superscripts when needed for clarity. Note that
the OU process has a well-known invariant measure $\Pi_{\theta}^{O}$,
which is a Gaussian measure with mean $\mu$ and variance $\sigma^{2}/2\lambda$,
while the CIR process has a well-known invariant measure $\Pi_{\theta}^{C}$,
which is a Gamma measure with shape parameter $2\lambda\mu/\sigma^{2}$
and rate parameter $2\lambda/\sigma^{2}$. Both these measures have
Lebesgue densities. 

In all simulations to follow, the relevant process is simulated over
the time interval $[0,1]$ with a step size $\delta=1/100$. Since
both the OU and CIR processes have known transition densities, we
simulate exact paths. All simulations are implemented in R on a 2012
Macbook Pro.


\section{Sampling $\nu$-Bridges}

Let $O$ and $C$ represent stationary solutions to (\ref{eq:ou-sde})
and (\ref{eq:cir-sde}), respectively. An easy way to validate the
correctness of the approximate and exact samplers proposed in \chapref{4}
is to simulate unconstrained sample paths of $O$ and $C$, and compare
the marginal distributions of these sample paths with generalized
bridges sampled according to \chapref{4} with endpoints drawn from
$O_{0,1}$ and $C_{0,1}$. Then, the marginal distributions of these
generalized bridges at any time $t\in[0,1]$ ought to be the same
as the distribution of the unconstrained sample path. 

For clarity, we fix in this section $\theta=1,\mu=0,\sigma=1$ for
$O$, and $\theta=1,\mu=1,\sigma=1$ for $C$. In order to sample
$O_{0,1}$- or $C_{0,1}$-bridges, we must be able to sample exactly
from $O_{0,1}$ and $C_{0,1}$. This is easy; we can simply sample
a path of $O$ or $C$, and take its values at times $t=\{0,1\}$
as an exact draw from $O_{0,1}$ or $C_{0,1}$ respectively.

With this in mind, we present the sample distributions of $O_{0,1}$-bridges
and $C_{0,1}$-bridges of $O$ and $C$ against the empirical distributions
of the unconstrained stationary solutions in \figref{joint-invariant}.
The M-H algorithm for the exact sampler is carried out with $M=10$
(the number of $T$-values simulated in order to calculate $\hat{\rho}$),
and no burn-in is used since the proposals are independent of one
another.

\begin{figure}[h]
\includegraphics[width=1\columnwidth]{/Users/marshall/Documents/senior/thesis/figures/invariant_densities}

\caption{\label{fig:joint-invariant}Q-Q plots comparing $10,000$ sample paths
at $t=1/2$ of unconstrained stationary Ornstein-Uhlenbeck (left)
and Cox-Ingersoll-Ross (right) processes to those of their $\nu$-bridges,
using an approximate (first row) and exact (second row) sampling scheme.}
\end{figure}


We first note that the approximate sampler produces marginal distributions
that are reasonable approximations to the exact sampler, though slightly
thinner-tailed than they should be. This is a result of the approximate
sampler drawing $\nu$-bridges conditioned on the bridges being hit
by a hitting diffusion; in this experiment, these hitting diffusions
tend towards the mean of the marginal initial and end distributions,
reducing the variance in-between. Though the exact samples appear
to be slightly thin-tailed as well, we will demonstrate in \tabref{KS-stats}
that we cannot reject the hypothesis that the exact samples and the
unconstrained diffusion are drawn from the same distribution at any
reasonable confidence level. We also present in \tabref{KS-stats}
analogous simulations for non-stationary solutions to (\ref{eq:ou-sde});
in particular, we consider solutions with initial distributions $X_{0}\sim\mathcal{N}(1,1/2)$,$X_{0}\sim2\mbox{Bern}-1$,
and $X_{0}\sim\mbox{Expo}(2)$. As before, we can draw exactly from
$O_{0,1}$ merely by simulating a solution to (\ref{eq:ou-sde}) with
the appropriate initial condition, and taking its value at times $t=\{0,1\}$. 

\tabref{KS-stats} confirms for a wide range of unorthodox joint initial/end
distributions, the exact sampler indeed samples from the true bridge
distribution at the cost of more computational power. We use $M=1$
for the exact sampler to minimize computational cost, which does not
seem to have a significant adverse impact on the quality of the exact
samples. We note that the computing time required to exactly sample
an OU process with initial distribution $\mathcal{N}(1,1/2)$ is far
higher than any other process considered; this is a result of the
lower probability of a hitting diffusion (which has an initial distribution
with mean $e^{-1}$ and mean-reverts towards $0$) intersecting with
the bridge. This relatively low probability of a hitting diffusion
intersection is reflected in the large K-S statistic for the approximately
sampled bridge, and speaks to the idea that to ``make up'' for a
poor approximate sampler, an exact sampler must use more computational
power. It is also worth noting that the K-S statistics presented in
\tabref{KS-stats} represent the most conservative estimates of the
deviation from true unconstrained diffusions\textemdash in particular,
the marginal distribution of the sampled bridges, whether approximate
or exact, are the same as the unconstrained diffusion at times $t=\{0,1\}$
by definition, and $t=1/2$ represents the point in time at which
the marginal distributions of these bridges have the most freedom
to drift from the true distribution.

\begin{table}
\begin{tabular}{@{}llllll@{}} \toprule \multirow{2}{*}{Process}                 & \multirow{2}{*}{Initial Condition} & \multicolumn{2}{l}{K-S stat. ($\times 10^{-2}$)} & \multicolumn{2}{l}{CPU (rel.)} \\ \cmidrule(l){3-4}                                           &                                    & Approx.                   & Exact                  & Approx.                  & Exact                 \\ \midrule \multicolumn{1}{l|}{\multirow{4}{*}{OU}} & $\mathcal{N}(1,1/2)$               & 4.38***                   & 1.79                   & 3.13                     & 35.6                  \\ \multicolumn{1}{l|}{}                    & $2\text{Bern}-1$                   & 2.80***                   & 1.81                   & 3.01                     & 12.0                  \\ \multicolumn{1}{l|}{}                    & $\text{Expo}(2)$                   & 2.75**                    & 2.13*                  & 2.91                     & 11.5                  \\ \multicolumn{1}{l|}{}                    & Stationary                         & 2.03*                     & 1.46                   & 2.87                     & 12.0                  \\                                          &                                    &                           &                        &                          &                       \\[-2.5ex] \multicolumn{1}{l|}{CIR}                 & Stationary                         & 2.41**                    & 1.34                   & 2.86                     & 10.8                  \\ \bottomrule \end{tabular} 

\begin{centering}
\begin{tabular}{lllll>{\raggedright}m{2cm}}
\multirow{1}{*}{} & \multirow{1}{*}{} &  &  &  & \tabularnewline
\end{tabular}
\par\end{centering}

\caption{\label{tab:KS-stats}Kolmogorov-Smirnov statistics comparing $10,000$
sample paths at $t=1/2$ of approximately and exactly sampled generalized
bridges against the corresponding unconstrained diffusions, as well
as the relative running times of sampling the bridges against sampling
the unconstrained diffusions.}
\end{table}


Somewhat unfortunately for our approximate sampler (which is roughly
an order of magnitude faster than our exact sampler), we are able
to reject the hypothesis that it samples from the unconstrained diffusion
at $t=1/2$ for every initial condition examined, at the $95\%$ confidence
level. However, recall that the MCEM algorithm proposed in \chapref{EM}
involves taking an expectation of an integral over an entire diffusion
bridge sample path, and so the K-S test on the marginal distribution
of our sampled paths at $t=1/2$ represents an overly stringent view
on the quality of these samples. In the next section, we shall experiment
with our MCEM algorithm and demonstrate that.....


\section{Inference on Distributional Data}

In this section, we verify the correctness of the MCEM algorithm proposed
in \chapref{EM}. 

As a first toy example, we consider stationary OU and CIR processes
with the parameters from the previous section (importantly, both processes
have unit diffusion coefficient). In particular, we Of course, we
know how to draw from the joint initial/end distributions of these
processes, and so an implementation of the MCEM algorithm 

\begin{table*}
\centering \begin{tabular}{@{}cllllllllll@{}} \toprule \multirow{2}{*}{Iteration} &  & \multicolumn{2}{l}{Approx.}                                           & \multicolumn{2}{l}{Exact}                                              \\ \cmidrule(lr){3-4}                             &  & \multicolumn{1}{c}{$\hat{\lambda}$} & \multicolumn{1}{c}{$\hat{\mu}$} & \multicolumn{1}{c}{$\hat{\lambda}$} & \multicolumn{1}{c}{$\hat{\mu}$}  \\ \cmidrule(r){1-1} \cmidrule(lr){3-6} 0                          &  & 2.00                                & 2.00                            & 2.00                                & 2.00                                                        \\ 1                          &  & 1.80                                & 0.21                            & 1.10                                & 0.38                                                        \\ 2                          &  & 1.45                                & 0.12                            & 1.04                                & 0.05                                                        \\ 3                          &  & 1.21                                & 0.04                            & 0.88                                & 0.03                                                          \\ 4                          &  & 1.26                                & 0.00                            & 1.02                                & 0.05                                                          \\ 5                          &  & 1.11                                & 0.00                            & 1.00                                & 0.04                                                          \\ 6                          &  & 1.07                                & 0.00                            & 0.95                                & 0.04                                                          \\ 7                          &  & 1.08                                & 0.00                            & 1.04                                & 0.02                                                          \\ 8                          &  & 1.08                                & 0.01                            & 0.98                                & 0.04                                                          \\ 9                          &  & 1.07                                & 0.01                            & 1.04                                & 0.02                                                          \\ 10                         &  & 1.07                                & 0.00                            & 1.03                                & 0.03                                                        \\ \bottomrule \end{tabular} 

\begin{centering}
\begin{tabular}{lllll>{\raggedright}m{2cm}}
\multirow{1}{*}{} & \multirow{1}{*}{} &  &  &  & \tabularnewline
\end{tabular}
\par\end{centering}

\caption{\label{tab:mcem-stationary}MCEM iterations for inference on the parameters
of an OU process with unit diffusion, given the joint distribution
at $t=\{0,1\}$ induced by a stationary solution to (\ref{eq:ou-sde}),
using approximate and exact generalized bridge samplers. The $i$th
iteration is performed using $100\times\left\lfloor i^{1.5}\right\rfloor $
MC samples, and the Fletcher-Reeves conjugate gradient method is used
for maximization.}
\end{table*}


\begin{sidewaystable}
\centering \begin{tabular}{@{}cllllllllllllllllllll@{}} \cmidrule(r){1-6} \cmidrule(l){8-21} \multirow{3}{*}{Iteration} &  & \multicolumn{4}{l}{$\mathcal{N}(0,1/2)$}                                                                                                      &  & \multicolumn{4}{l}{$2\text{Bern}-1$}                                                                                      &  & \multicolumn{4}{l}{$\text{Expo}(2)$}                          &  & \multicolumn{4}{l}{$\hat{\mathcal{N}}(0,1/2)$ (100 samples)}                      \\ \cmidrule(lr){3-6} \cmidrule(lr){8-11} \cmidrule(lr){13-16} \cmidrule(l){18-21}                             &  & \multicolumn{2}{l}{Approx.}                                           & \multicolumn{2}{l}{Exact}                                             &  & \multicolumn{2}{l}{Approx.}                                           & \multicolumn{2}{l}{Exact}                         &  & \multicolumn{2}{l}{Approx.}   & \multicolumn{2}{l}{Exact}     &  & \multicolumn{2}{l}{Approx.}                       & \multicolumn{2}{l}{Exact}     \\ \cmidrule(lr){3-4} \cmidrule(lr){8-9} \cmidrule(lr){13-14} \cmidrule(lr){18-19}                            &  & \multicolumn{1}{c}{$\hat{\lambda}$} & \multicolumn{1}{c}{$\hat{\mu}$} & \multicolumn{1}{c}{$\hat{\lambda}$} & \multicolumn{1}{c}{$\hat{\mu}$} &  & \multicolumn{1}{c}{$\hat{\lambda}$} & \multicolumn{1}{c}{$\hat{\mu}$} & \multicolumn{1}{c}{$\hat{\lambda}$} & $\hat{\mu}$ &  & $\hat{\lambda}$ & $\hat{\mu}$ & $\hat{\lambda}$ & $\hat{\mu}$ &  & \multicolumn{1}{c}{$\hat{\lambda}$} & $\hat{\mu}$ & $\hat{\lambda}$ & $\hat{\mu}$ \\ \cmidrule(r){1-1} \cmidrule(lr){3-6} \cmidrule(lr){8-11} \cmidrule(lr){13-16} \cmidrule(l){18-21}  0                          &  & 2.00                                & 2.00                            & 2.00                                & 2.00                            &  & 2.00                                & 2.00                            & 2.00                                & 2.00        &  & 2.00            & 2.00        & 2.00            & 2.00        &  & 2.00                                & 2.00        & 2.00            & 2.00        \\ 1                          &  & 1.89                                & 0.46                            & 1.30                                & 0.15                            &  & 0.95                                & 1.23                            & 0.79                                & -0.03       &  & 0.94            & 0.09        & 1.04            & -0.17       &  & 1.23                                & 0.36        & 0.99            & 0.04        \\ 2                          &  & 1.71                                & 0.25                            & 1.23                                & 0.14                            &  & 1.10                                & 1.06                            & 0.83                                & -0.09       &  & 1.10            & -0.04       & 1.27            & -0.01       &  & 1.10                                & 0.05        & 1.04            & 0.04        \\ 3                          &  & 1.46                                & 0.16                            & 1.19                                & 0.02                            &  & 1.25                                & 1.05                            & 1.11                                & 0.01        &  & 1.17            & -0.04       & 1.18            & 0.01        &  & 1.09                                & 0.03        & 1.02            & 0.03        \\ 4                          &  & 1.55                                & 0.17                            & 1.12                                & 0.01                            &  & 1.17                                & 1.02                            & 1.06                                & 0.05        &  & 1.25            & 0.02        & 1.17            & 0.04        &  & 1.15                                & 0.02        & 0.99            & 0.03        \\ 5                          &  & 1.45                                & 0.13                            & 1.07                                & 0.11                            &  & 1.17                                & 1.03                            & 1.03                                & 0.01        &  & 1.17            & 0.00        & 1.03            & -0.03       &  & 1.14                                & 0.06        & 1.02            & 0.14        \\ 6                          &  & 1.39                                & 0.12                            & 1.12                                & 0.09                            &  & 1.15                                & 0.97                            & 1.03                                & 0.01        &  & 1.17            & 0.00        & 1.07            & -0.01       &  & 1.07                                & 0.06        & 1.00            & 0.09        \\ 7                          &  & 1.42                                & 0.13                            & 1.15                                & 0.05                            &  & 1.10                                & 0.99                            & 1.07                                & -0.00       &  & 1.15            & 0.00        & 1.00            & -0.02       &  & 1.10                                & 0.07        & 0.99            & 0.06        \\ 8                          &  & 1.45                                & 0.12                            & 1.07                                & 0.06                            &  & 1.11                                & 0.98                            & 1.07                                & 0.00        &  & 1.10            & 0.00        & 1.25            & 0.04        &  & 1.07                                & 0.08        & 1.01            & 0.04        \\ 9                          &  & 1.33                                & 0.10                            & 1.08                                & 0.00                            &  & 1.09                                & 0.98                            & 1.01                                & 0.00        &  & 1.18            & -0.01       & 1.10            & 0.04        &  & 1.10                                & 0.09        & 1.05            & 0.02        \\ 10                         &  & 1.28                                & 0.08                            & 1.07                                & 0.03                            &  & 1.03                                & 0.99                            & 1.01                                & 0.01        &  & 1.17            & 0.01        & 1.04            & -0.00       &  & 1.09                                & 0.09        & 1.01            & 0.05        \\ \bottomrule \end{tabular}

\begin{centering}
\begin{tabular}{lllll>{\raggedright}m{2cm}}
\multirow{1}{*}{} & \multirow{1}{*}{} &  &  &  & \tabularnewline
\end{tabular}
\par\end{centering}

\caption{\label{tab:diff-diffusions}MCEM iterations for inference on the parameters
of an OU process with unit diffusion given the joint distribution
at $t=\{0,1\}$ induced by a solution to (\ref{eq:ou-sde}) with various
initial conditions and $\lambda=1,\mu=0$, using approximate and exact
generalized bridge samplers. The $i$th iteration is performed using
$100\times\left\lfloor i^{1.5}\right\rfloor $ MC samples, and the
Fletcher-Reeves conjugate gradient method is used for maximization.}
\end{sidewaystable}

\end{document}
